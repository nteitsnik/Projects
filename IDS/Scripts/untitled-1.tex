\documentclass{beamer}

% Theme selection:
\usetheme{Madrid} % Choose from: Berlin, Warsaw, Frankfurt, CambridgeUS, etc.

% Optional packages
\usepackage{graphicx} % For inserting images
\usepackage{hyperref} % For hyperlinks
\usepackage{lipsum}   % For dummy text (optional)

% Title info
\title{My First LaTeX Beamer Presentation}
\author{Your Name}
\institute{Your Institution}
\date{\today}

\begin{document}

% Title slide
\begin{frame}
  \titlepage
\end{frame}

% Table of contents
\begin{frame}
  \frametitle{Outline}
  \tableofcontents
\end{frame}

\section{Introduction}
\begin{frame}
  \frametitle{Introduction}
  This is your first Beamer presentation. \\
  You can add sections, frames, images, equations and more!
\end{frame}

\section{Basic Slide with Bullet Points}
\begin{frame}
  \frametitle{What You Can Do}
  \begin{itemize}
    \item Write presentation slides
    \item Add bullet lists
    \item Include figures and equations
    \item Export to PDF
  \end{itemize}
\end{frame}

\section{Adding an Image}
\begin{frame}
  \frametitle{Sample Image}
  \begin{center}
    \includegraphics[width=0.6\textwidth]{example-image} % Requires example-image.png in the same folder
  \end{center}
\end{frame}

\section{Hyperlinks and References}
\begin{frame}
  \frametitle{Useful Resources}
  You can add clickable links:

  \begin{itemize}
    \item \href{https://www.latex-project.org/}{LaTeX project website}
    \item \href{https://overleaf.com}{Overleaf: Online LaTeX Editor}
  \end{itemize}
\end{frame}

\section{Conclusion}
\begin{frame}
  \frametitle{Conclusion}
  \begin{enumerate}
    \item Beamer is powerful for academic presentations
    \item You can create slides using logical LaTeX structure
    \item Customize themes and layout freely
  \end{enumerate}
\end{frame}

% Thank you slide
\begin{frame}
  \frametitle{Thanks!}
  Thank you for viewing this Beamer presentation.
  
  \vspace{1cm}
  Questions?
\end{frame}

\end{document}